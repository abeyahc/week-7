\documentclass{report}
\usepackage{array}

\input{preamble}
\input{macros}
\input{letterfonts}

\usepackage{mathtools}

\title{\Huge{Discrete Mathematics}\\Week 7}
\author{\huge{Abeyah Calpatura}}
\date{}

\begin{document}
\maketitle
\section*{5.1}
\subsection*{Exercises} \\
\text{Abeyah Calpatura} \\
\#2, 13, 21, 25, 30, 42, 45, 54, 72, 76
\\
\renewcommand{\arraystretch}{1.5}
\textbf{\#2} 
\sol{ 
    \begin{align*}
        & \text{$b_1=\frac{5-1}{5+1}=\frac{4}{6}=\frac{2}{3}$} \\
        & \text{$b_2=\frac{5-2}{5+2}=\frac{3}{7}$} \\
        & \text{$b_3=\frac{5-3}{5+3}=\frac{2}{8}=\frac{1}{4}$} \\
        & \text{$b_4=\frac{5-4}{5+4}=\frac{1}{9}$} \\
        & \text{answer = $\frac{2}{3}$, $\frac{3}{7}$, $\frac{1}{4}$, $\frac{1}{4}$}
    \end{align*}
}

\textbf{\#13}
\sol{ $1-\frac{1}{2}, \frac{1}{2}-\frac{1}{3}, \frac{1}{3}-\frac{1}{4}, \frac{1}{4}-\frac{1}{5}, \frac{1}{5}-\frac{1}{6}, \frac{1}{6}-\frac{1}{7}$
    \begin{align*}
            \begin{tabular}{ c c }
            n & $a_n$ \\
            1 & $a_1=1-\frac{1}{2}=\frac{1}{1}-\frac{1}{1+1}$ \\ 
            2 & $a_2=\frac{1}{2}-\frac{1}{3}=\frac{1}{2}-\frac{1}{2+1}$ \\
            3 & $a_3=\frac{1}{3}-\frac{1}{4}=\frac{1}{3}-\frac{1}{3+1}$ \\
            4 & $a_4=\frac{1}{4}-\frac{1}{5}=\frac{1}{4}-\frac{1}{4+1}$ \\
            5 & $a_5=\frac{1}{5}-\frac{1}{6}=\frac{1}{5}-\frac{1}{5+1}$ \\
            6 & $a_6=\frac{1}{6}-\frac{1}{7}=\frac{1}{6}-\frac{1}{6+1}$ \\
            \end{tabular}
    \end{align*}
    \begin{align*}
        \text{general answer = $a_n=\frac{1}{n}-\frac{1}{n+1}, n\ge1$}
    \end{align*}
}

\textbf{\#21}
\sol{
    \begin{align*}
        & \text{$\sum_{k=1}^{3}(k^2+1)$} \\
        & \text{$=(1^2+1)+(2^2+1)+(3^2+1)$} \\ 
        & \text{$=2+5+10$} \\
        & \text{$=17$}
    \end{align*}
}

\textbf{\#25}
\sol{
    \begin{align*}
        & \text{$\displaystyle \prod_{k=2}^{2}(1-\frac{1}{k})$} \\
        & \text{$=(1-\frac{1}{2})=\frac{1}{2}$} \\
        & \text{answer = $\frac{1}{2}$}
    \end{align*}
}

\textbf{\#30}
\sol{
    \begin{align*}
        & \text{$\sum_{j=1}^{n}j(j+1)$} \\
        & \text{$=1(1+1)+2(2+1)+3(3+1)+\cdots+n(n+1)$} \\
    \end{align*}
}

\textbf{\#42}
\sol{ By separating off the final term, we have
    \begin{align*}
        & \text{$\sum_{m=1}^{n+1}m(m+1)$} \\
        & \text{$=\sum_{m=1}^{n}m(m+1)+(n+1)((n+1)+1)$} \\
        & \text{$=\sum_{m=1}^{n}m(m+1)+(n+1)(n+2)$}
    \end{align*}
}

\textbf{\#45}
\sol{ $(2^2-1)\cdot(3^2-1)\cdot(4^2-1)$
    \begin{align*}
        & \text{$\displaystyle \prod_{k=2}^{4}(k^2-1)$} \\
    \end{align*}
}

\textbf{\#54}
\sol{ $i=k+1$
    \begin{align*}
        & \text{$\displaystyle \prod_{n}^{k=1}\frac{k}{k^2+4}$} \\
        & \text{$\displaystyle \prod_{n}^{k=1}\frac{(k+1)-1}{((k+1)-1)^2+4}$} \\ 
        & \text{$\displaystyle \prod_{n+1}^{i=2}\frac{i-1}{(i-1)^2+4}$}
    \end{align*}
}

\textbf{\#72}
\sol{
    \begin{align*}
    & \text{$\begin{pmatrix} 7 \\ 4 \end{pmatrix}$} \\
    & \text{$=\frac{7!}{4!(7-4)!}$} \\
    & \text{$=\frac{7!}{4!3!}$} \\
    & \text{$=\frac{7\cdot6\cdot5\cdot4!}{4!\cdot3\cdot2\cdot1}$} \\
    & \text{$=\frac{7\cdot6\cdot5}{3\cdot2\cdot1}$} \\
    & \text{$=35$}
    \end{align*}
}

\textbf{\#76}
\sol{
    \begin{align*}
    & \text{$\begin{pmatrix} n+1 \\ n-1 \end{pmatrix}$} \\
    & \text{$=\frac{(n+1)!}{(n-1)!(n+1-(n-1))!}$} \\
    & \text{$=\frac{(n+1)!}{(n-1)!\cdot 2!}$} \\
    & \text{$=\frac{(n+1)\cdot n \cdot (n-1)!}{(n-1)!2}$} \\
    & \text{$=\frac{n(n+1)}{2}$}
    \end{align*}
}

\newpage

\section*{5.2}
\subsection*{Exercises} \\
\text{Abeyah Calpatura} \\
\#3, 4, 7
\\

\textbf{\#3}
\sol{ For every positive integer $n$, $P(n)$ represents the formula: \\
    \begin{align*}
        & \text{$1^2+2^2+\dots +n^2=\frac{n(n+1)(2n+1)}{6}$} \\ \\
        & \textbf{a.} \; \text{Is P(1) true?} \\
        & \text{$1^2=\frac{1(1+1)(2\cdot1+1)}{6}$} \\
        & \text{$1=\frac{1\cdot2\cdot3}{6}$} \\
        & \text{$1=1$} \\
        & \text{Answer: Yes, $P(1)$ is true.} \\ \\
        & \textbf{b. $P(k)$} \\
        & \text{Answer: $1^2+2^2+\dots +k^2=\frac{k(k+1)(2k+1)}{6}$} \\ \\
        & \textbf{c. $P(k+1)$} \\
        & \text{$1^2+2^2+\dots +k^2+(k+1)^2=\frac{(k+1)(k+1+1)(2(k+1)+1)}{6}$} \\
        & \text{$\frac{k(k+1)(2k+1)}{6}+(k+1)^2=\frac{(k+1)(k+2)(2k+3)}{6}$} \\ \\
        & \textbf{d. $P(k+1)$} \\
        & \text{Answer: $P(k+1)$ is true assuming that $P(k)$ is true.}
    \end{align*}
}

\newpage

\textbf{\#4}
\sol{ For each integer n with $n\ge1$, $P(n)$ represents the formula: \\
    \begin{align*}
        & \text{$\sum_{i=1}^{n-1}i(i+1)=\frac{n(n-1)(n+1)}{3}$} \\ \\
        & \textbf{a.} \; \text{Is P(2) true?} \\
        & \text{$\sum_{i=1}^{2-1}i(i+1)=\frac{2(2-1)(2+1)}{3}$} \\
        & \text{$1(1+1)=\frac{2(1)(3)}{3}$} \\
        & \text{$2=\frac{2(3)}{3}$} \\
        & \text{$2=2$} \\
        & \text{Answer: Yes, $P(2)$ is true.} \\ \\
        & \textbf{b. $P(k)$} \\
        & \text{Answer: $\sum_{i=1}^{k-1}i(i+1)=\frac{k(k-1)(k+1)}{3}$} \\ \\
        & \textbf{c. $P(k+1)$} \\
        & \text{$\sum_{i=1}^{k}i(i+1)=\frac{(k+1)(k)(k+2)}{3}$} \\
        & \text{$\frac{k(k-1)(k+1)}{3}+k(k+1)=\frac{(k+1)(k)(k+2)}{3}$} \\ \\
        & \textbf{d. $P(k+1)$} \\
        & \text{Answer: $P(k+1)$ is true assuming that $P(k)$ is true.}
    \end{align*}
}

\newpage

\textbf{\#7}
\sol{ For every integer $n\geq 1$, 
    \begin{align*}
        & \text{$1+6+11+16+ \dots +(5n-4)= \frac{n(5n-3)}{2}$} \\ \\
        & \textbf{a.} \; \text{Is P(1) true?} \\
        & \text{$1=\frac{1(5\cdot1-3)}{2}$} \\
        & \text{$1=\frac{1(2)}{2}$} \\
        & \text{$1=1$} \\
        & \text{Answer: Yes, $P(1)$ is true.} \\ \\
        & \textbf{b. $P(k)$} \\
        & \text{Answer: $1+6+11+16+ \dots +(5k-4)= \frac{k(5k-3)}{2}$} \\ \\
        & \textbf{c. $P(k+1)$} \\
        & \text{$1+6+11+16+ \dots +(5k-4)+(5(k+1)-4)= \frac{(k+1)(5(k+1)-3)}{2}$} \\
        & \text{$\frac{k(5k-3)}{2}+(5(k+1)-4)= \frac{(k+1)(5k+2)}{2}$} \\ \\
        & \textbf{d. $P(k+1)$} \\
        & \text{Answer: $P(k+1)$ is true assuming that $P(k)$ is true.}
    \end{align*}
}

\newpage

\section*{5.3}
\subsection*{Exercises} \\
\text{Abeyah Calpatura} \\
\#3, 5, 9, 11
\\ \\
\textbf{3.} Stamps are sold in packages containing either 5 stamps or 8 stamps. \\ \\
\textbf{\#3a}
\sol{ Show that a person can obtain 5, 8, 10, 13, 15, 16, 20, 21, 24, or 25 stamps by buying a collectino of 5-stamp packages and 8-stamp packages. \\
    \begin{align*} 
    \begin{tabular}{ | c | c |}
    \hline
    Number of stamps & How to obtain it \\
    \hline
    5 & 5 \\
    \hline
    8 & 8 \\
    \hline
    10 & $5+5$ \\
    \hline
    13 & $5+8$ \\
    \hline
    15 & $5+5+5$ \\
    \hline
    16 & $8+8$ \\
    \hline
    20 & $5+5+5+5$ \\
    \hline
    21 & $5+8+8$ \\
    \hline
    24 & $8+8+8$ \\
    \hline
    25 & $5+5+5+5+5$ \\
    \hline
    \end{tabular}
    \end{align*} \\
}
\textbf{\#3b}
\sol{ Use mathematical induction to show that any quantity of at least 28 stamps can be obtained
    \begin{align*}
        & \text{Show that P(28) is true.} \\
        & \text{Answer: $28=5+5+5+5+8$} \\
        & \text{Show that P(k) is true.} \\
        & \text{Answer: $k=5a+8b$} \\
        & \text{Show that P(k+1) is true.} \\
        & \text{Answer: $k+1=5a+8b$}
    \end{align*}
}

\newpage

\textbf{\#5}
\sol{ For each positive number $n$, let $P(n)$ be the inequality \\
    \begin{align*}
        & \text{$2^n < (n+1)!$} \\ \\
        & \textbf{a.} \; \text{Is P(2) true?} \\
        & \text{$2^2 < (2+1)!$} \\
        & \text{$4 < 3!$} \\
        & \text{$4 < 6$} \\
        & \text{Answer: Yes, $P(2)$ is true.} \\ \\
        & \textbf{b. $P(k)$} \\
        & \text{Answer: $2^k < (k+1)!$} \\ \\
        & \textbf{c. $P(k+1)$} \\
        & \text{$2^{k+1} < (k+2)!$} \\
        & \text{$2\cdot2^k < (k+2)(k+1)!$} \\
        & \text{$2\cdot2^k < (k+2)!$} \\ \\
        & \textbf{d. $P(k+1)$} \\
        & \text{Answer: $P(k+1)$ is true assuming that $P(k)$ is true.}
    \end{align*}
}

\textbf{\#9}
\sol{ By mathematica induction, prove
    \begin{align*}
        & \text{$7^n-1$ is divisible by 6, for each integer $n \ge 0$} \\ \\
        & \textbf{a.} \; \text{Is P(0) true?} \\
        & \text{$7^0-1$} \\
        & \text{$1-1$} \\
        & \text{$0$} \\
        & \text{Answer: Yes, $P(0)$ is true.} \\ \\
        & \textbf{b. $P(k)$} \\
        & \text{$7^k-1=6m$} \\
        & \text{Answer: $7^k-1$ is divisible by 6} \\ \\
        & \textbf{c. $P(k+1)$} \\
        & \text{$7^{k+1}-1=7\cdot 7^k -1$} \\
        & \text{$=7\cdot [(7^k -1)+1] -1$} \\
        & \text{$=7\cdot [6m +1]-1$} \\
        & \text{$=42m+7-1$} \\
        & \text{$=42m+6$} \\
        & \text{$=6(7m+1)$} \\ \\
        & \textbf{d. $P(k+1)$} \\
        & \text{Answer: $P(k+1)$ is true assuming that $P(k)$ is true.}
    \end{align*}
} 

\textbf{\#11}
\sol{ By mathematical induction, prove
    \begin{align*}
        & \text{$3^{2n}-1$ is disivible by 8, for each integer $n \ge 0$} \\ \\
        & \textbf{a.} \; \text{Is P(0) true?} \\
        & \text{$3^{2\cdot0}-1$} \\
        & \text{$3^0-1$} \\
        & \text{$0$} \\
        & \text{Answer: Yes, $P(0)$ is true.} \\ \\
        & \textbf{b. $P(k)$} \\
        & \text{$3^{2k}-1=8m$} \\
        & \text{Answer: $3^{2k}-1$ is divisible by 8} \\ \\
        & \textbf{c. $P(k+1)$} \\
        & \text{$3^{2(k+1)}-1=3^2\cdot3^{2k}-1$} \\
        & \text{$=9\cdot(3^{2k}-1)+8$} \\
        & \text{$=9\cdot8m+8$} \\
        & \text{$=8(9m+1)$} \\ \\
        & \textbf{d. $P(k+1)$} \\
        & \text{Answer: $P(k+1)$ is true assuming that $P(k)$ is true.}
    \end{align*}
}
\end{document}